\documentclass{beamer}

\begin{document}

\begin{frame}
  \frametitle{Gaussian Graphical Models}
  Let $\mathcal G := (V, E)$ be a graph consisting of 
  \begin{itemize}
    \item vertices $V := v_1, \ldots v_n$, and 
    \item edges $E := e_1, \ldots e_k$.
  \end{itemize}
  We can also represent $G$ as an adjacency matrix $\mathcal A := (a_{ij})_{i,j  = 1}^n$.
  \vfill 
  In a \alert{Gaussian Graphical Model} (GGM), each node $v_i$ is then associated with a random variable $x_i$, and the assumption is made that 
  \[
    X := (x_1, \ldots x_n)' \sim \mathcal N(\mu, \Sigma),
  \]
  where, for $i\ne j$, $\mathcal A_{ij} = 0 \implies (\Sigma)_{ij} = 0$.
\end{frame}

\end{document}
